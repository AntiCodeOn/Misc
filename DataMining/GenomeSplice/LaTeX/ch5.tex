\section{Rezultati analize}
\label{ch:ch5}


Za procjenu uspješnosti klasifikacije koriste se različite mjere. Za razumijevanje rezultata potrebno je razumjeti i problem koji se modelom opisuje i pokusava rjesiti.Klasifikator koji bi uvijek davao klasu N (nije detektirana granica IE ili EI) bi i dalje imao veliku točnost, budući da je većina genoma sastavljena od nekodirajućih dijelova.

\begin{itemize}
   \item Točnost 
   \item TPR
   \item FPR
   \item F-mjera
   \item AUC
   \item ROC
\end{itemize}


Tablica prikazuje uspješnost algoritama korištenjem gore opisanih mjera.
