\section{Naslov poglavlja}
\label{ch:ch1}


U jednoj stanici organizama nalaze se informacija potrebna za razvoj cjelokupne jedinke. Kod većine organizama ta informacija kodirana je u molekuli koja se zove deoksiribonukleinska kiselina  (DNK) uz izuzetak nekih organizama koji koriste ribonukleinsku kiselinu (RNK). 

Ribonukleinska kiselina (RNK) je polinukleotidna jednolančana molekula i nastaje transkripcijom komplementarnih baza gena iz molekule DNK. Ovaj se proces odvija pod utjecajem enzima RNK polimeraza koji prepoznaju karakteristično mjesto (promotor) na DNK molekuli gdje započinju proces prepisivanja korištenjem ribonukleotidnih molekula – adenin (A) se prepisuje u uracil (U), timin (T) u adenin, guanin (G) se prepisuje u citozin (C) te obrnuto, citozin u guanin. Ovaj se proces odvija dok enzim ne stigne do karakterističnog niza zvanog terminator. U analogiji s komunikacijskom teorijom, DNK služi kao izvor informacije [ref shannon]l, RNK molekula je signal, a prijemnik su ribosomi u citoplazmi koji dekodiraju ovu poruku I na osnovu nje proizvode molekule proteina.

