\section{Stabla odlučivanja}
\label{ch:ch3}

Stabla odlučivanja vrlo su popularna metoda dubinske analize podataka zbog svoje jednostavnosti i brzine. Algoritam koristi “podijeli-pa-vladaj” paradigmu i najjednostavnije se može opisati u rekurzivnom obliku. Najprije se odabire atribut koji će biti korijen stabla. Za svaku moguću vrijednost koju taj atribut može imati izvodi se jedna grana. Na ovaj način se skup podataka dijeli u podskupove. Ako jedan od čvorova stabla dobivenih na ovaj način sadrži samo instance jedne klase, zaustavlja se rekurzivni postupak. Inače, nastavljamo dijeliti preostale podskupove prema novim atributima. Odabir atributa nije proizvoljan, želimo odabrati atribute na takav način da stablo bude što manje tako da naš algoritam radi brže. 

Indukcijski zadatak provodi se nad skupom objekata koji su opisani kolekcijom atributa. Svaki atribut mjeri neko važno svojstvo objekta te u pravilu poprima veličinu iz diskretnog skupa vrijednosti. Objekti pripadaju jednoj od dvije ili više međusobno isključivih klasa. Iz skupa podataka za trening u kojem su klase objekata poznate izvode se klasifikacijska pravila. Ukoliko su atributi adekvatni (objekti s istim vrijednostima atributa pripadaju istoj klasi) uvijek je moguće konstruirati stablo odlučivanja koje će ispravno klasificirati sve objekte u trening skupu\cite{Witten01}. Međutim, stablo odlučivanja koje ispravno klasificira samo trening skup nije korisno jer zapravo samo izražava podatke koje imamo u tablici (trening skupa) u obliku stabla. Kako bi se  konstruirano stablo moglo primjeniti i na buduće, dotad neviđene podatke (testni skup) stablo odlučivanja mora sadržavati smislene informacije o odnosima atributa objekta i klase tog objekta. 

Računalni znanstvenik John Ross Quinlan dao je značajan doprinos razvoju algoritama za dubinsku analizu temeljenim na stablima odlučivanja\cite{Wu01}\cite{Quinlan02}. Istraživanje je potaknuto potrebom da se unaprijede algoritmi sa sposobnošću pronalaska znanja u samim skupovima podataka bez korištenja domenskih eksperata. Prvi iz niza varijanti stabla odlučivanja koje je dizajnirao, ID3, je dizajniran za skupove podataka s velikim brojem objekata i velikim brojem atributa objekta poštujući ograničenje da konstruirano stablo odlučivanja bude jednostavno (kako bi se ubrzao proces generiranja takvog stabla uz minimalne računalne resurse). Posljedica ovakvih ograničenja u dizajnu je da ID3 ne garantira globalno optimalno stablo odlučivanja\cite{Quinlan02}.
