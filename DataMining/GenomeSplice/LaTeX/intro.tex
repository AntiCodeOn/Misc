\section*{Uvod}
\label{ch:intro}
\addcontentsline{toc}{section}{Uvod}

Početak 21. stoljeća donio je sa sobom i kraj Projekta humanog genoma. Točno pedeset godina nakon što je otkrivena struktura DNK molekule\cite{Watson01}, u travnju 2003. godine, Međunarodni konzorcij za sekvenciranje ljudskog genoma objavio je uspješan dovršetak sekvenciranja nešto više od tri milijarde nukleotidnih baza - koliko ih sadrži ljudski DNK. Pod okriljem ovog projekta sekvencirani su i genomi drugih eukariotskih organizama\footnote{Kvasac \textit{Saccharomyces cerevisiae}, oblić \textit{Caenorhabditis elegans}, vinska mušica \textit{Drosophila melanogaster}, domaći miš \textit{Mus musculus} i biljka \textit{Arabidopsis thaliana} } te stotine vrsta bakterija i arheja. Sekvenciranje novih vrsta nastavljeno je i nakon dovršetka projekta\cite{Cox01}. 
\par
Budući da se radi o ogromnim količinama podataka, vrlo brzo postalo je jasno kako ih je moguće smisleno interpretirati samo uz pomoć odgovarajućeg sofvera. Kao posljedica toga, počelo se razvijati interdisciplinarno područje bioinformatike čiji je cilj kombiniranjem znanja iz područja molekularne biologije i računalnih znanosti te podataka o genomu dovesti do novih spoznaja o načinu na koji funkcionira živi svijet na Zemlji. Jedan od primjera primjene računalne znanosti, detekcija granica kodirajućih i nekodirajućih dijelova gena, obrađen je u ovom seminaru.

